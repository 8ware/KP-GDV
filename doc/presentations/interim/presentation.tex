\documentclass[
	ddcfooter,
	german,
%	nototalpages,
%	handout
]{tudbeamer}

\usepackage[utf8]{inputenc}

\usepackage[babel, german=guillemets]{csquotes}
\usepackage{xspace}

\usepackage{listings}

\usepackage[style=numeric-verb, backend=bibtex]{biblatex}
\DeclareFieldFormat{urldate}{\space\tiny[#1]}
\addbibresource{../../references}

\setbeamerfont{caption}{size=\tiny}

\graphicspath{ {./graphics/} }

\einrichtung{Fakultät Informatik}
\institut{Institut für Software- und Multimediatechnik}
\professur{Computergraphik und Visualisierung}

\title{Zwischenpräsentation}
\subtitle{KP Graphische Datenverarbeitung WS 14/15}
\author{Felix Mai}
\date{26. Januar 2015}

\newcommand{\blankline}{\newline\space\newline}
\newcommand{\arrow}{\ensuremath{\rightarrow}~}
\newcommand{\thus}{\item[\arrow]}
\newcommand{\zb}{z.\,B.\xspace}

\newcommand{\code}[1]{\texttt{#1}}


\begin{document}

\maketitle

\begin{frame}[t]
	\frametitle*{Gliederung}
	\tableofcontents
\end{frame}



\section{Einleitung}

\subsection{Aufgabenstellung}

\begin{frame}{Einleitung}{Aufgabenstellung}
\end{frame}

\subsection{Motivation}

\begin{frame}{Einleitung}{Motivation}
\end{frame}

\subsection{Setup}

\begin{frame}{Einleitung}{Setup}
\end{frame}

\subsection{Teilprobleme}

\begin{frame}{Einleitung}{Teilprobleme}
\end{frame}



\section{Jaco-Arm}

\begin{frame}{Jaco-Arm}
\end{frame}



\section{Kinect}

\begin{frame}{Kinect}
\end{frame}



\section{Ausblick}

\begin{frame}{Ausblick}
\end{frame}



\begin{frame}[t]
	\vspace*{2.5cm}
	\centerline{\huge{Demonstration des}}
	\centerline{\huge{aktuellen Standes}}
\end{frame}


\begin{frame}[t]
	\vspace*{2.5cm}
	\centerline{\huge{Fragen oder Anmerkungen?}}
\end{frame}


\section*{References}

\begin{frame}[t]
	\frametitle*{References}
	\renewcommand*{\bibfont}{\tiny}
	\nocite{*}
	\printbibliography[title=References]
\end{frame}

\end{document}

